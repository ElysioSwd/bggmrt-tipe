%Définition du type de document ainsi que de la langue%
\documentclass[12pt]{report}
\usepackage[top= 4cm, bottom=4cm, left=2cm, right=2cm, a4paper]{geometry}
\usepackage{luatextra}
\usepackage[french]{babel}

%Def titre auteur date%
\title{TIPE - Anticipation des phénomènes \\météorologiques en mer}
\author{Julien \textsc{Brohan} - Théo \textsc{Gandy} - Eyal \textsc{Gros} \\ Louis \textsc{Maley} -
Mattéo \textsc{Riedinger} - Matthieu \textsc{Touré}}
\date{Juin 2019}

%Paramètres pour la mise en forme de la table des matières, et la création de liens%
\usepackage{hyperref}
\usepackage{url}
\renewcommand{\thesection}{\Roman{section})}
\renewcommand{\thesubsection}{~~\Roman{section}-\arabic{subsection})}
\renewcommand{\baselinestretch}{1.12}

%Apport du paquet pour la création d'équations%
\usepackage{amsmath}
\usepackage{amsfonts}

%Apport des paquets pour la création de tableaux%
\usepackage{hhline}
\usepackage{multirow}
\usepackage{caption}

%Paquets nécessaire à la personnalisation des en-têtes/pieds de page%
\usepackage{fancyhdr}
\pagestyle{fancy}
\usepackage{lastpage}

%Apport du paquet pour l'ajout de couleurs%
\usepackage{color}

%Apport du paquet pour l'ajout de code en langage C%
\usepackage{listings}
\lstset{language=C}

%Appel des commandes pour faciliter la mise en forme des en-têtes/pieds de page%
\makeatletter
\let\thetitle\@title
\let\theauthor\@author
\let\thedate\@date
\makeatother

%Personnalisation des en-têtes%
\fancyhead[L]{}
\fancyhead[C]{\thetitle}
\fancyhead[R]{}

%Personnalisation des pieds de page%
\fancyfoot[L]{}
\fancyfoot[C]{\thedate}
\fancyfoot[R]{\thepage /\pageref{LastPage}}

%Définition de la largeur des en-têtes/pieds de page%
\renewcommand{\headrulewidth}{1pt}
\renewcommand{\footrulewidth}{1pt}



%DÉBUT DU DOCUMENT ICI%

\begin{document}

\maketitle
\setcounter{page}{0}
\tableofcontents
\thispagestyle{empty}
\clearpage

\section*{Introduction}

    Les tempêtes en milieu maritime sont les plus grandes menaces pour les bateaux de tous types. 
    En effet, une fois pris dans une tempête il est impossible d’en sortir par soi-même. 
    Les tempêtes représentent la première cause de perte de marchandise en mer allant de simples 
    pertes de conteneurs mal attachés, au naufrage du bateau en question. 
    Selon le World Shipping Council, le nombre de conteneurs perdus en mer varie entre 2 600 et 10 000
    chaque année, c’est à dire entre 7 et 27 par jour et peut-être même plus puisque la déclaration des
    conteneurs perdus n’est pas obligatoire.\\


    Ils y a très peu d’accidents graves de plaisanciers suite à des conditions météorologiques compliquées. Cependant, naviguer dans de bonnes conditions est un confort et bien que certains sites / applications fournissent de très bonnes données, recalculer une trajectoire en prenant en compte ces données se fait souvent manuellement. 
    Pour une navigation dans de bonnes conditions il est donc impératif d’être bien préparé face aux
    intempéries et un accès facile tout public à ces informations est nécessaire. De plus, en cas de
    prévisions météorologiques fausses il est important de pouvoir avertir les autres navigateurs, le plus rapidement possible.\\
    Voici les principaux phénomènes météorologiques à risques rencontrés en mer :\\


    \begin{itemize}
        \item Le vent ressenti localement à proximité du bateau qui peut avoir pour origine les grandes masses d'air océaniques et continentales (vent synoptique) et/ou une origine locale due au relief et à l'ensoleillement (brise).
        \item La houle créée par un vent (synoptique), dont on arrive à posséder des informations précises.
        \item La mer du vent levée par un vent local (comme la brise) qui peut changer brutalement et dont les prévisions sont moins précises.
        \item Les orages en mer, difficiles à prévoir, apparaissant souvent en fin d'après-midi. Ils sont accompagnés de fortes rafales de vent changeant et de grains réduisant la visibilité.
        \item La brume (visibilité réduite) qui peut se former rapidement et insidieusement à tout moment du jour et de la nuit au-dessus de la mer.\\
    \end{itemize}


    Existerait-t-il donc un moyen permettant aux navigateurs d’être averti en temps réel des intempéries avec une possibilité de signaler par la même occasion, d’éventuelles prévisions erronées ?\\

    %====================== PLAN A INSÉRER AU FUR ET A MESURE ==========================%
    Dans un premier temps, nous définirons les différentes données dont nous avons besoin, et quelles sont leurs spécificités.

    \clearpage

%===================== PREMIÈRE PARTIE : DONNÉES ( TITRE À DÉFINIR ) =====================%
\section{Données météorologiques}

    %------------TEMPÊTES------------%
    \subsection{Les différentes intempéries}

        Les tempêtes sont des phénomènes météorologiques récurrents et présent partout sur le globe.


    %-----------Types de données----------%
    \subsection{Quelles données prendre, sous quelle forme}
    
        Plusieurs types de données sont nécessaires afin d'établir une prédiction météorologique. Il faut des mesures de température, de vitesse du vent et de pression atmosphérique.
     

%======================= LIENS =======================%
    \newpage

\bibliographystyle{plain}
\begin{thebibliography}{9}

    \bibitem{Sources de donnees meteo-oceaniques:}
    Sources de donnees meteo-oceaniques\\
    \url{https://archimer.ifremer.fr/doc/1992/rapport-988.pdf}\\

    \bibitem{source possible pour carte 1:}
    source possible pour carte 1:\\
    \url{https://www.bbc.com/weather/map }    \\

    \bibitem{source possible pour carte 2:}
    source possible pour carte 2:\\
    \url{https://www.ventusky.com/?p=10;-169;1&l=temperature-2m} \\

    \bibitem{openweathermap = map ultra cool mais pas facile :}
    openweathermap = map ultra cool mais pas facile :\\
    \url{https://home.openweathermap.org/} \\
    \url{https://openweathermap.org/api/weather-map-2}\\

    \bibitem{source possible pourtempêtes}
    source possible pour tempêtes :\\
    \url{https://ocean.si.edu/planet-ocean/waves-storms-tsunamis/hurricanes-typhoons-and-cyclones}\\

    \bibitem{data utilisée pour la météo:}
    data utilisée pour la météo:\\
    \url{https://courses.lumenlearning.com/geophysical/chapter/collecting-weather-data/}\\


    \bibitem{étapes des prévisions:}
    étapes des prévisions:\\
    \url{http://www.meteofrance.fr/prevoir-le-temps/la-prevision-du-temps/les-etapes-de-prevision}\\

\end{thebibliography}
\end{document}
